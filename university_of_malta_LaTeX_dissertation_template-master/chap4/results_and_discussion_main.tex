\chapter{Results \& Discussion}
%\textbf{Should include a reiteration of the experiments, and their outcome.  Together with a description (discussion).  Preamble should include a reminder of the aims and objectives together with a list of experiments to achieve these.  Should include many charts and other visualization with appropriate descriptions}.

\section{Data Collection Validation}

In the pre-processing experiment, the method used for collecting and filtering the 1000 Tweets.
To test that the distribution and collection was being done correctly, figures~\ref{fig:preprocessdist} and~\ref{fig:finaldist} where generated.


\begin{figure}[!htb]
\minipage{0.5\textwidth}
  \includegraphics[width=\linewidth]{2021-00-01 Filtered Language Distribution.png}
  \caption[Pre-Process Filtered Language Distribution]{ }
  \label{fig:preprocessdist}
\endminipage\hfill
\minipage{0.5\textwidth}
  \includegraphics[width=\linewidth]{2021-01-01 Filtered Language Distribution - Final.png}
  \caption[Final Filtered Language Distribution]{ }
  \label{fig:finaldist}
\endminipage
\end{figure}


\section{Pre-Processing Effect}

The compound sentiment scores returned from the \ac{VADER} model was averaged and plotted for the 3 dates used in the experiment.
The 4 graphs produced, figures~\ref{fig:EnglishPre} to~\ref{fig:GermanPre}, show a time series graph of each language over the 3 days which are 1 month apart.
The mean scores where also summed and the difference between the pre-processed and non-processed total was of 0.17\%.

\begin{figure}[!htb]
\minipage{0.5\textwidth}
  \includegraphics[width=\linewidth]{English Process VS NotProcessed.png}
  \caption[English Process VS NotProcessed]{ }\label{fig:EnglishPre}
\endminipage\hfill
\minipage{0.5\textwidth}
  \includegraphics[width=\linewidth]{Spanish Process VS NotProcessed.png}
  \caption[Spanish Process VS NotProcessed]{ }\label{fig:SpanishPre}
\endminipage
\end{figure}
\begin{figure}[!htb]
\minipage{0.5\textwidth}
  \includegraphics[width=\linewidth]{French Process VS NotProcessed.png}
  \caption[French Process VS NotProcessed]{ }\label{fig:FrenchPre}
\endminipage\hfill
\minipage{0.5\textwidth}
  \includegraphics[width=\linewidth]{German Process VS NotProcessed.png}
  \caption[German Process VS NotProcessed]{ }\label{fig:GermanPre}
\endminipage
\end{figure}

\noindent The difference calculated along with the shapes of the graphs plotted were not considered significant enough to remove pre-processing.
However the pre-process function was amended to keep more features like emojis and hashtags, which prior to this experiment where being removed.

\section{Daily Twitter Mean Sentiment}

Just as with figures~\ref{fig:EnglishPre} to~\ref{fig:GermanPre}, time series graphs where plotted for the mean sentiment scores on the larger 180,000 tweet dataset.
When all the languages are plotted against each other in figure~\ref{fig:globalall} show no clear trend, however some languages are closer to each other than others.
When these languages are separated in figures~\ref{fig:globalmean} and~\ref{fig:globaleu} the mean of the compound scores can be seen to related more to each other.

\begin{figure}[h!]
\includegraphics[scale=0.3]{Daily Mean All.png}
\caption[Daily Mean All]{ }
\label{fig:globalall}
\end{figure}

\noindent The effect of translation is very apparent with the reduced information given from the lines plotted from non-English tweets.

\begin{figure}[h!]
\includegraphics[scale=0.3]{Daily Mean Global.png}
\caption[Daily Mean Global]{ }
\label{fig:globalmean}
\end{figure}

\noindent Although not followed exactly, the shape of the Spanish line corresponds to the peaks and dips of the English line.

\begin{figure}[h!]
\includegraphics[scale=0.3]{Daily Mean Europe.png}
\caption[Daily Mean Europe]{ }
\label{fig:globaleu}
\end{figure}

\noindent The line plots for these languages where expected to be more closely related however some similarities can be seen.
On holidays such as 1st January (new years) or 14th February(valentines day) most countries experience a peak in positivity.
Inversely on the 14th March every country experienced a dip in positivity.

\section{Daily Twitter Sentiment Classification}



\section{Daily Article vs Twitter Mean Sentiment}



\section{Daily Twitter Sentiment Classification with Article Mean Sentiment}



\section{Word Frequency in the Form of Word Clouds}

