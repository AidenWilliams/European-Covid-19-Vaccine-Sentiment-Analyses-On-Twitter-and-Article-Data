\chapter{Background \& Literature Overview}

In December 2020 the European Commission published a survey titled: Public Opinion on Covid-19 Vaccination in the European Union 2020~\citep{eupublicopinion}.
In this survey, a total of 24,424 interviews where conducted in over 27 European countries to see the public's views on:

\begin{enumerate}
  \item Attitudes to vaccination.
  \item General satisfaction with EU measures to fight the pandemic.
  \item Information on Covid-19.
\end{enumerate}
\noindent The data collected was done manually via either online forms or telephone calls and was represented visually in the published survey.
Similar in scope to this \ac{IAPT}, \citet{ztrk2018} performed an extensive sentiment analysis using related Turkish and English tweets on the Syrian refugee crisis.

%A large dataset of Twitter statuses available online, is maintained by Panacea Lab~\citep{banda2020largescale}.
%This dataset spans a range between January 2020 till now, June 2021 as it is updated every two weeks.
%On average there are 4 million tweets written in various languages.
%For analyses and getting the \ac{SA} scores the \ac{VADER} model is commonly used.

\section{Data Collection}
\index{Data Collection|(}

For a proper analyses of online european content, a varied dataset is collected.
Two general sets of sources, social media platforms and online newspapers are identified as sufficient and adequate for this \ac{IAPT}.

\subsection{Social Media Platform Data Collection}
\index{Data Collection!Social Media Platform Data Collection}

Social media platform data collection can be done via several methods.
Most platforms offer official \ac{REST} \ac{API}, while for others a traditional web-scraping techniques are used to collect the data.
In 2016 Twitter users posted an average of 500 million tweets, daily~\citep{Crannell2016}.
In the same year, the number of active Twitter users exceeded 22\% of the internet users in the world~\citep{Kayser2016}.
This amounted to 342 million daily active users at the time.
More recently this figure has changed to 330 million daily users ~\citep{tankovska_2021}.
Another popular social media platform is Facebook.
In the case of Malta 92.3\% of the population ~\citep{napoleoncat} is an active user of the platform.
The possibility of using Facebook as a data source was explored where Covid-19 related influencers, social leaders and local media pages' posts would be collected as well as their audience user engagement, .i.e comments and shares.
However the \ac{API} was found to not be as accessible or usable as the Twitter \ac{API}.
Due to its global reach and usable \ac{API} it was decided that Twitter would be the social media platform from which public opinionated data will be collected.

\subsection{Newspaper Article Data Collection}
\index{Data Collection!Newspaper Article Data Collection}

Unlike social media platforms the more traditional newspapers, sometimes referred to as the fourth estate, are decentralised and not limited to one platform.
In fact many modern newspapers utilize social media to share and advertise on.
During the research done in this \ac{IAPT} it was found that most published works that use an article headline dataset, have this dataset collected manually such as in \citet{newspaper_headlines}.
However a number of online \ac{API}s where found and tested with the intention of collecting a dataset spanning 6 months.
These \ac{API}s were \citet{newsapi} and \citet{newscatcher}, notwithstanding their topic and country search functionality, accessing articles past one month was a paid feature.
So it was decided to not use these \ac{API}s due to budgeting reasons.
It was then decided that the article heading dataset would be collected and validated manually.

\index{Data Collection|)}

\section{Data Processing}
\index{Data Processing|(}
\blindtext
\subsection{Translation}
\index{Data Processing!Translation}
\blindtext
\subsection{Pre-Processing}
\index{Data Processing!Pre-Processing}
\blindtext
\index{Data Processing|)}


\section{Sentiment Analyses}
\index{Sentiment Analyses|(}
\blindtext
\subsection{\ac{VADER}}
\index{Sentiment Analyses!\ac{VADER}}
\blindtext
\index{Sentiment Analyses|)}

%\section{Evaluation Criteria}
%This section should contain information on the metrics and background used to evaluate your work.
%
%\section{Related Work}
%\textbf{In this section you need to explain (and reference) similar work in literature}.  Make sure to:
%
%\begin{itemize}
% \item Give a systematic overview of papers with related/similar work
% \item Highlight similarities/differences to your work (perhaps in the form of a table)
%\end{itemize}
%
%Note that this section may be sectioned based on the different aspects of your dissertation.  Some referenced text, as an example \citep{Arrighi2003, WithersMartinez2012, Ebejer2016}.

%
%
%\chapter{Background \& Literature Review}
%%\textbf{In this section you need to explain all the theory required to understand your dissertation (i.e.\ the following chapters)}
%
%\section{Background}
%\textbf{cite things}
%In December 2020 the European Commission published a survey titled: Public Opinion on Covid-19 Vaccination in the European Union 2020~\citep{eupublicopinion}.
%In this survey, a total of 24,424 interviews where conducted in over 27 European countries to see the public's views on:
%
%\begin{enumerate}
%  \item Attitudes to vaccination.
%  \item General satisfaction with EU measures to fight the pandemic.
%  \item Information on Covid-19.
%\end{enumerate}
%
%\noindent The data collected was done manually via either online forms or telephone calls and was represented visually in the published survey.
%
%A large dataset of Twitter statuses available online, is maintained by Panacea Lab~\citep{banda2020largescale}.
%This dataset spans a range between January 2020 till now, June 2021 as it is updated every two weeks.
%On average there are 4 million tweets written in various languages.
%
%For analyses and getting the \ac{SA} scores the \ac{VADER} model is commonly used.
%
%%\index{Some Technique One|(}
%%\blindtext
%%\subsection{Some Sub-technique One}
%%\blindtext
%%\index{Some Technique One!Some Sub-technique One}
%%\blindtext
%%\subsubsection{Some Sub-sub-technique One}
%%\blindtext
%%\index{Some Technique One!Some Sub-sub-technique One}
%%\blindtext
%%\index{Some Technique One|)}
%%
%%\section[Some Technique Two]{Some Technique Two with Super Long Title Which Will Overrun In Header}
%%\index{Some Technique Two|(}
%%\blindtext[5]
%%
%%Imagine some colourful description on Some Technique Three\index{Some Technique Three}.
%%
%%\index{Some Technique Two|)}
%%
%%\section{Evaluation Criteria}
%%This section should contain information on the metrics and background used to evaluate your work.
%
%\section{Literature Review}
%
%
%
%%\textbf{In this section you need to explain (and reference) similar work in literature}.  Make sure to:
%%
%%\begin{itemize}
%% \item Give a systematic overview of papers with related/similar work
%% \item Highlight similarities/differences to your work (perhaps in the form of a table)
%%\end{itemize}
%%
%%Note that this section may be sectioned based on the different aspects of your dissertation.  Some referenced text, as an example \citep{Arrighi2003, WithersMartinez2012, Ebejer2016}.
%%
%%\section{An Example of Suppressing Page Numbers on A Float Page}
%%
%%Refer to Figure~\ref{fig:largegoku}.
%%
%%\begin{figure}[!ht]
%%	\thisfloatpagestyle{empty} %% This is the key line to suppress page decorations (including nos.) on float pages.
%%	\centering
%%	\includegraphics[width=0.9\textwidth]{goku-large}
%%	\caption[Short Random Caption]{\blindtext}
%%	\label{fig:largegoku}
%%\end{figure}
%%
%%\blindtext
%%
%%\section{Summary}
%%\blindtext
