\chapter{Background \& Literature Overview}



\section{Some Technique One}
\index{Some Technique One|(}
\blindtext
\subsection{Some Sub-technique One}
\blindtext
\index{Some Technique One!Some Sub-technique One}
\blindtext
\subsubsection{Some Sub-sub-technique One}
\blindtext
\index{Some Technique One!Some Sub-sub-technique One}
\blindtext
\index{Some Technique One|)}

\section[Some Technique Two]{Some Technique Two with Super Long Title Which Will Overrun In Header}
\index{Some Technique Two|(}
\blindtext[5]

Imagine some colourful description on Some Technique Three\index{Some Technique Three}.

\index{Some Technique Two|)}

\section{Evaluation Criteria}
This section should contain information on the metrics and background used to evaluate your work.

\section{Related Work}
\textbf{In this section you need to explain (and reference) similar work in literature}.  Make sure to:

\begin{itemize}
 \item Give a systematic overview of papers with related/similar work
 \item Highlight similarities/differences to your work (perhaps in the form of a table)
\end{itemize}

Note that this section may be sectioned based on the different aspects of your dissertation.  Some referenced text, as an example \citep{Arrighi2003, WithersMartinez2012, Ebejer2016}.

%
%
%\chapter{Background \& Literature Review}
%%\textbf{In this section you need to explain all the theory required to understand your dissertation (i.e.\ the following chapters)}
%
%\section{Background}
%\textbf{cite things}
%In December 2020 the European Commission published a survey titled: Public Opinion on Covid-19 Vaccination in the European Union 2020~\citep{eupublicopinion}.
%In this survey, a total of 24,424 interviews where conducted in over 27 European countries to see the public's views on:
%
%\begin{enumerate}
%  \item Attitudes to vaccination.
%  \item General satisfaction with EU measures to fight the pandemic.
%  \item Information on Covid-19.
%\end{enumerate}
%
%\noindent The data collected was done manually via either online forms or telephone calls and was represented visually in the published survey.
%
%A large dataset of Twitter statuses available online, is maintained by Panacea Lab~\citep{banda2020largescale}.
%This dataset spans a range between January 2020 till now, June 2021 as it is updated every two weeks.
%On average there are 4 million tweets written in various languages.
%
%For analyses and getting the \ac{SA} scores the \ac{VADER} model is commonly used.
%
%%\index{Some Technique One|(}
%%\blindtext
%%\subsection{Some Sub-technique One}
%%\blindtext
%%\index{Some Technique One!Some Sub-technique One}
%%\blindtext
%%\subsubsection{Some Sub-sub-technique One}
%%\blindtext
%%\index{Some Technique One!Some Sub-sub-technique One}
%%\blindtext
%%\index{Some Technique One|)}
%%
%%\section[Some Technique Two]{Some Technique Two with Super Long Title Which Will Overrun In Header}
%%\index{Some Technique Two|(}
%%\blindtext[5]
%%
%%Imagine some colourful description on Some Technique Three\index{Some Technique Three}.
%%
%%\index{Some Technique Two|)}
%%
%%\section{Evaluation Criteria}
%%This section should contain information on the metrics and background used to evaluate your work.
%
%\section{Literature Review}
%
%
%
%%\textbf{In this section you need to explain (and reference) similar work in literature}.  Make sure to:
%%
%%\begin{itemize}
%% \item Give a systematic overview of papers with related/similar work
%% \item Highlight similarities/differences to your work (perhaps in the form of a table)
%%\end{itemize}
%%
%%Note that this section may be sectioned based on the different aspects of your dissertation.  Some referenced text, as an example \citep{Arrighi2003, WithersMartinez2012, Ebejer2016}.
%%
%%\section{An Example of Suppressing Page Numbers on A Float Page}
%%
%%Refer to Figure~\ref{fig:largegoku}.
%%
%%\begin{figure}[!ht]
%%	\thisfloatpagestyle{empty} %% This is the key line to suppress page decorations (including nos.) on float pages.
%%	\centering
%%	\includegraphics[width=0.9\textwidth]{goku-large}
%%	\caption[Short Random Caption]{\blindtext}
%%	\label{fig:largegoku}
%%\end{figure}
%%
%%\blindtext
%%
%%\section{Summary}
%%\blindtext
