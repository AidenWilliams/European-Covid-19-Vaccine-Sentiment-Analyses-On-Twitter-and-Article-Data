\chapter{Background \& Literature Overview}

\section{Background}

The Covid-19 pandemic, also known as the coronavirus pandemic, is an ongoing global pandemic first identified in December 2019 in Wuhan, China.
As of 12 June 2021, more than 175 million cases have been confirmed, with more than 3.78 million confirmed deaths attributed to COVID-19, making it one of the deadliest pandemics in history.
Late in 2020 vaccination programs started to roll out in various countries.
In January 2021, countries that form part of the European union started their vaccine programs.
In December 2020 the European commission published a survey titled: Public Opinion on Covid-19 Vaccination in the European Union 2020~\citep{eupublicopinion}.
In this survey, a total of 24,424 interviews where conducted in over 27 European countries to see the public's views on:

\begin{enumerate}
  \item Attitudes to vaccination.
  \item General satisfaction with EU measures to fight the pandemic.
  \item Information on Covid-19.
\end{enumerate}
\noindent The data collected was done manually via either online forms or telephone calls and was represented visually in the published survey.

\subsection{Data Collection}
\index{Data Collection|(}

For a proper analyses of online european content, a varied dataset is collected.
Two general sets of sources, social media platforms and online newspapers are identified as sufficient and adequate for this \ac{IAPT}.

\subsubsection{Social Media Platform Data Collection}
\index{Data Collection!Social Media Platform Data Collection}

Social media platform data collection can be done via several methods.
Most platforms offer official \ac{REST} \ac{API}, while for others a traditional web-scraping techniques are used to collect the data.
In 2016 Twitter users posted an average of 500 million tweets, daily~\citep{Crannell2016}.
In the same year, the number of active Twitter users exceeded 22\% of the internet users in the world~\citep{Kayser2016}.
This amounted to 342 million daily active users at the time.
More recently this figure has changed to 330 million daily users ~\citep{tankovska_2021}.

Another popular social media platform is Facebook.
In the case of Malta 92.3\% of the population ~\citep{napoleoncat} is an active user of the platform.
The possibility of using Facebook as a data source was explored where Covid-19 related influencers, social leaders and local media pages' posts would be collected as well as their audience user engagement, .i.e comments and shares.
However the \ac{API} was found to not be as accessible or usable as the Twitter \ac{API}.
Due to its global reach and usable \ac{API} it was decided that Twitter would be the social media platform from which public opinionated data will be collected.
A large scale database of tweets is maintained by \citet{banda2020largescale}.

\subsubsection{Newspaper Article Data Collection}
\index{Data Collection!Newspaper Article Data Collection}

Unlike social media platforms the more traditional newspapers, sometimes referred to as the fourth estate, are decentralised and not limited to one platform.
In fact many modern newspapers utilize social media to share and advertise on.
During the research done for this \ac{IAPT} it was found that most published works that use an article headline dataset, have this dataset collected manually such as in \citet{newspaper_headlines}.
However a number of online \ac{API}s where found and tested with the intention of collecting a dataset spanning 6 months.
These \ac{API}s were newsapi~\citep{newsapi}) and newscatcher~\citep{newscatcher}, notwithstanding their topic and country search functionality, accessing articles past one month was a paid feature.
So it was decided to not use these \ac{API}s due to budgeting reasons.
It was then decided that the article heading dataset would be collected and validated manually.

\index{Data Collection|)}

\subsection{Data Processing}
\index{Data Processing|(}

Europe is a diverse continent full of different peoples, cultures and languages.
To match data with an origin country two methods where used.
Either the data came with a geo-tag, .i.e the origin country was specified somewhere with the text of the data.
Or the source language was used.
Due to a number of different languages collected the decision had to be taken whether to implement a multi-lingual \ac{SA} model or use machine translation.
\citet{Balahur2014} state that commercial engines are able to translate from and into a large number of languages.
\citet{Arajo2020} use and experiment number commercially available machine translation engines, produced the table~\ref{tab:translators}.
From table~\ref{tab:translators} the Google and Yandex \ac{API}s where checked due to them achieving the highest Macro-F1 score.
Between the two the Google Cloud Translation \ac{API} was cheapest.

\begin{table*}[ht!]\centering
\ra{1.3}
\begin{tabular}{rrr}\toprule
$Translation Method$ & $Macro-F1$ & $Applicability$\\ \midrule
Google& 0.73±(0.02)& 0.47±(0.04)&\\
Yandex& 0.73±(0.02)& 0.54±(0.03)&\\
Bing& 0.72±(0.02)& 0.53±(0.03)&\\
Baseline& 0.70±(0.02)& 0.51±(0.03)&\\
\bottomrule
\end{tabular}
\caption[Macro-F1 and Applicability mean () results on different Translation Methods]{Macro-F1 and Applicability mean () results given each machine translation system among all language datasets when translated to English.}\label{tab:translators}
\end{table*}
\index{Data Processing|)}

\subsection{Sentiment Analyses}
\index{Sentiment Analyses|(}

As an active research field that has emerged recently, \ac{SA} is a discipline that extracts people’s feelings, opinions, thoughts and behaviors from user’s text data using \ac{NLP} methods~\citep{danneman2014social}.
Moreover, \ac{SA} is also known as opinion mining, with emphasis on text classification problem.
Extracting sentiment information from web-scale text data can be very challenging and expensive task due to large amount of data~\citep{FernndezGavilanes2016}.
A survey conducted by \citet{Lo2016} explores two main approaches for \ac{SA}, subjectivity and polarity detection.
There subjectivity detection described as an understanding on whether the content contains personal views and opinions as opposed to factual information.
And polarity detection as the study of subjectivity with different polarities, intensities or rankings.
In the context of this \ac{IAPT} polarity detection makes more sense than subjectivity detection.
A frequently used and readily accessible \ac{SA} model is \ac{VADER} ~\citep{Hutto_Gilbert_2014}.

%\subsection{The VADER Model}
%\index{Sentiment Analyses!The VADER Model}

\index{Sentiment Analyses|)}

\section{Literature Review}

\ac{SA} has been used several times to analyze and evaluate public opinions on a myriad of topics.
Similar in scope to this \ac{IAPT}, \citet{ztrk2018} performed an extensive sentiment analysis using related Turkish and English tweets on the Syrian refugee crisis.
\citet{MANSOUR201895} analyzes Twitter users from multiple countries for their responses to terrorism.
Based on 36 million tweets collected from Twitter, \citet{wang2012system} proposed a real-time \ac{SA} system for classification of political tweets during 2012 US presidential elections.
Their model achieved 59\% accuracy in predicting the sentiments of political tweets.
Other research on political issues include~\citep{park2016expanding, ahmed20162014, bollen2011twitter, cheong2011microblogging}

Twitter data has also been used for medical issue analyses.
Because the content generated on social media platforms extends to more personal topics such as the personal experience of going through cancer.
\citet{Crannell2016} found that cancer patients describe and explain their feelings about their diseases openly and candidly on Twitter.
\citet{lampos2010flu} use the Twitter \ac{API} and its geo-tagging feature to collect around 56 million UK tweets.
With this data Lampos et al, designed a model that was able to detect outbreaks of the flu in a specific area of the UK\@.
In 2017 \citet{elkin2017network} propose a system of equations and procedures that can be translated to apply to any social media data, other contagious diseases and geographies to mine large data sets for predicting future outbreaks.
Other research on medical issues include~\citep{corley2010text, culotta2010towards, broniatowski2013national}

Apart from Twitter data this \ac{IAPT} also explores news article headings.
In 2013 a report by \citet{GarciaDiego2013SdR} uses the financial news section from the New York Times of newspapers from 1905 to 2005.
Garcia successfully proves how the predictability of stock returns using news' content is concentrated in recessions.
\citet{RameshbhaiChaudharyJashubhai2019Omon} uses newspaper headlines for their \ac{SA} model comparison stating that multiple research has been done in opinion mining for online blogs, Twitter, Facebook, etc.
\citet{GhasiyaPiyush2021ICNA} analyze a database of more than 100 thousand news headlines and articles for knowledge regarding the Covid-19 pandemic, stating that:

\begin{displayquote}
Newspapers are providing rich information to the public about various issues such as the discovery of a new strain of coronavirus, lockdown and other restrictions and government policies.
\end{displayquote}
